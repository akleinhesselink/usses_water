\documentclass[12pt,]{article}
\usepackage{lmodern}
\usepackage{amssymb,amsmath}
\usepackage{ifxetex,ifluatex}
\usepackage{fixltx2e} % provides \textsubscript
\ifnum 0\ifxetex 1\fi\ifluatex 1\fi=0 % if pdftex
  \usepackage[T1]{fontenc}
  \usepackage[utf8]{inputenc}
\else % if luatex or xelatex
  \ifxetex
    \usepackage{mathspec}
  \else
    \usepackage{fontspec}
  \fi
  \defaultfontfeatures{Ligatures=TeX,Scale=MatchLowercase}
\fi
% use upquote if available, for straight quotes in verbatim environments
\IfFileExists{upquote.sty}{\usepackage{upquote}}{}
% use microtype if available
\IfFileExists{microtype.sty}{%
\usepackage{microtype}
\UseMicrotypeSet[protrusion]{basicmath} % disable protrusion for tt fonts
}{}
\usepackage[margin=1in]{geometry}
\usepackage{hyperref}
\PassOptionsToPackage{usenames,dvipsnames}{color} % color is loaded by hyperref
\hypersetup{unicode=true,
            colorlinks=true,
            linkcolor=black,
            citecolor=Blue,
            urlcolor=black,
            breaklinks=true}
\urlstyle{same}  % don't use monospace font for urls
\usepackage{graphicx,grffile}
\makeatletter
\def\maxwidth{\ifdim\Gin@nat@width>\linewidth\linewidth\else\Gin@nat@width\fi}
\def\maxheight{\ifdim\Gin@nat@height>\textheight\textheight\else\Gin@nat@height\fi}
\makeatother
% Scale images if necessary, so that they will not overflow the page
% margins by default, and it is still possible to overwrite the defaults
% using explicit options in \includegraphics[width, height, ...]{}
\setkeys{Gin}{width=\maxwidth,height=\maxheight,keepaspectratio}
\IfFileExists{parskip.sty}{%
\usepackage{parskip}
}{% else
\setlength{\parindent}{0pt}
\setlength{\parskip}{6pt plus 2pt minus 1pt}
}
\setlength{\emergencystretch}{3em}  % prevent overfull lines
\providecommand{\tightlist}{%
  \setlength{\itemsep}{0pt}\setlength{\parskip}{0pt}}
\setcounter{secnumdepth}{0}
% Redefines (sub)paragraphs to behave more like sections
\ifx\paragraph\undefined\else
\let\oldparagraph\paragraph
\renewcommand{\paragraph}[1]{\oldparagraph{#1}\mbox{}}
\fi
\ifx\subparagraph\undefined\else
\let\oldsubparagraph\subparagraph
\renewcommand{\subparagraph}[1]{\oldsubparagraph{#1}\mbox{}}
\fi

%%% Use protect on footnotes to avoid problems with footnotes in titles
\let\rmarkdownfootnote\footnote%
\def\footnote{\protect\rmarkdownfootnote}

%%% Change title format to be more compact
\usepackage{titling}

% Create subtitle command for use in maketitle
\newcommand{\subtitle}[1]{
  \posttitle{
    \begin{center}\large#1\end{center}
    }
}

\setlength{\droptitle}{-2em}
  \title{}
  \pretitle{\vspace{\droptitle}}
  \posttitle{}
  \author{}
  \preauthor{}\postauthor{}
  \date{}
  \predate{}\postdate{}

\usepackage{lineno}
\linenumbers
\usepackage{setspace}
\usepackage{todonotes}
\doublespacing
\usepackage{rotating}
\usepackage{color, soul}
\usepackage[font={normalsize},labelfont={bf},labelsep=quad]{caption}
\usepackage{sectsty}
\usepackage{bm,mathrsfs}
\usepackage{mathptmx}

\begin{document}

\renewcommand\linenumberfont{\normalfont\tiny\sffamily\color{gray}}

\definecolor{blue}{rgb}{0,0,0.7} \newcommand{\new}{\textcolor{blue}}

\begin{singlespace}

\begin{centering}

\Large{\textbf{Ecosystem and community resistance to six years of drought and deluge in the Sagebrush Steppe}}

\bigskip{} \bigskip{}

\renewcommand*{\thefootnote}{\fnsymbol{footnote}}

\normalsize{Andrew T. Tredennick\textsuperscript{1}, Peter B. Adler\textsuperscript{1}, and Andrew R. Kleinhesselink\textsuperscript{2}}

\bigskip{}

\textit{\small{\textsuperscript{1}Department of Wildland Resources and the Ecology Center, Utah State University, Logan, Utah 84322}}
\textit{\small{\textsuperscript{2}Department of Ecology and Evolutionary Biology, University of California, Los Angeles, Los Angeles, California xxxxx}}

\end{centering}

\vspace{3em}

Last compile: \today

\end{singlespace}

\begin{abstract}
Summarize it here
\vspace{2em}
\end{abstract}

\setlength{\parindent}{5ex}

\subsection{Introduction}\label{introduction}

Start it here

\subsection{Methods}\label{methods}

\subsubsection{Study Area}\label{study-area}

We conducted our precipitation manipulation experiment at the United
States Sheep Experimental Station (USSES) near Dubois, Idaho
(44.2\(^{\circ}\) N, 112.1\(^{\circ}\) W), 1500 m above sea level. The
vegetation is typical of high elevation sagebrush steppe. The plant
community is dominated by the shrub \emph{Artemesia tripartita} and
three perennial bunchgrasses, \emph{Pseudoroegneria spicata}, \emph{Poa
secunda}, and \emph{Hesperostipa comata}. During the period of our
experiment, average mean annual precipitation was xxx mm and mean
monthly temperature ranged from -x\(^{\circ}\)C in January to
x\(^{\circ}\)C in July.

\subsubsection{Precipitation Experiment}\label{precipitation-experiment}

Between 1926 and 1932, range scientists at the USSES established 26
permanent 1 m\(^2\) quadrats to track vegetation change over time. In
2007, we (well, one of us {[}P. Adler{]}) relocated 14 of the original
quadrats, six of which were inside a large, permanent livestock
exclosure. We use these six plots as control plots that have recieved no
treatment, just ambient precipitation. In spring 2011, we (well, two of
us {[}A. Kleinhesselink and P. Adler{]}) established 16 new 1 m\(^2\)
plots. We avoided areas on steep hill slopes, areas with greater than
20\% cover of bare rock, and areas with greater than 10\% cover of the
shrubs \emph{Purshia tridentata} and/or \emph{Amelanchier utahensis}. We
established the new plots in pairs and randomly assigned each plot in a
pair to receive a ``drought'' or ``irrigation'' treatment.

Drought and irrigation treatments were designed to decrease and increase
the amount of ambient precipitation by 50\%, respectively. To achieve
this, we used a system of rain-out shelters and automatic irrigation
(Gherardi \& Sala 2013). The rain-out shelters consisted of transparent
acrylic shingles 1-1.5 m above the ground that covered an area of 2.5 by
2 m. The shingles intercepted approximately 50\% of incoming rainfall,
which was channeled into 75 liter containers. Captured rainfall was then
pumped out of the containers and sprayed on to the adjacent irrigation
plot via two suspended sprinklers. Pumping was triggered by float
switches once water levels reached about 20 liters. We disconnected the
irrigation pumps each fall and reconnected them, often with difficulty,
each spring. The rain-out shelters remained in place throughout the
year.

To make sure the treatments were having the desired effects, we
monitored soil moisture in four of the drought-irrigation pairs using
Decagon Devices (Pulman, Washington) 5TM and EC-5 soil moisture sensors.
We installed four sensors in each plot, two at 5 cm soil depth and two
at 25 cm soil depth. We also installed four sensors in areas nearby the
four selected plot pairs to measure ambient soil moisture at the same
depths. Soil moisture measurements were automatically logged every four
hours. We coupled this temporally intensive soil moisture sampling with
spatially extensive readings taken at six points within all 16 plots and
associated ambient measurement areas. These snapshot data were collected
on 06/06/2012, 04/29/2015, 05/07/2015, 06/09/2015, and 05/10/2016 using
a handheld EC-5 sensor.

\subsubsection{Data Collection}\label{data-collection}

\paragraph{Aboveground Net Primary
Productivity}\label{aboveground-net-primary-productivity}

We estimated aboveground net primary productivity (ANPP) using a
radiometer to related ground reflectance to plant biomass (see Byrne et
al. 2011 for a review).

\subsection{Acknowledgments}\label{acknowledgments}

This work was funded by the National Science Foundation through grants
DEB-1353078 and DEB-1054040 to PBA, a Graduate Research Fellowship to
ARK, and a Postdoctoral Resaerch Fellowship in Biology to ATT
(DBI-1400370). Additional support came from the Utah Agricultural
Experiment Station (journal paper xxxx). We thank the many summer
research technicians who collected the data reported in this paper.

\singlespace
\setlength{\parindent}{0ex}

\subsection*{References}\label{references}
\addcontentsline{toc}{subsection}{References}

\hypertarget{refs}{}
\hypertarget{ref-Byrne2011}{}
Byrne, K.M., Lauenroth, W.K., Adler, P.B. \& Byrne, C.M. (2011).
Estimating Aboveground Net Primary Production in Grasslands: A
Comparison of Nondestructive Methods. \emph{Rangeland Ecology \&
Management}, 64, 498--505.

\hypertarget{ref-Gherardi2013}{}
Gherardi, L.A. \& Sala, O.E. (2013). Automated rainfall manipulation
system: a reliable and inexpensive tool for ecologists.
\emph{Ecosphere}, 4, 1--10.


\end{document}
